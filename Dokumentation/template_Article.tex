\documentclass[]{article}
\usepackage[ngerman]{babel}

%opening
\title{Erfahrungen aus der Erstellung des Projektes \glqq ThermoAmpel\grqq{} für das Seminar \glqq Mikrocontrollerschaltungen - Realisierung in Hard- und Software\grqq}
\author{Ilhan Aydin, Lev Perschin}

\begin{document}

\maketitle
\newpage
\begin{abstract}

\end{abstract}
\newpage
\section{Erste Erfahrungen/Schritte}
Wir, Ilhan und Lev, hatten wenig Vorwissen in diesem Bereich. Wir haben noch nie ein Schaltplan erstellt, die Komponenten auf dem Board verteilt, es gedruckt, bestückt und anschließend programmiert. Das Wissen, welches wir hatten, waren Kenntnisse aus der Vorlesung \glqq Technische Informatik\grqq{} und ähnliche.


\section{Schaltplan}
\begin{itemize}
	\item Auswahl der richtigen Teile - größtenteils von PDF übernommen
	\item Berechnung von LED Widerstanden
	\item Übersichtlichkeit wichtig
	\item Offene Pins mit einem Stecker(?) versehen, damit man sie in Zukunft leichter verwenden kann
\end{itemize}
\section{Board}
\begin{itemize}
	\item Positionierung der Teile - im Nachhinein vielleicht wie es am schlausten wäre?
	\item VDD Leitung etwas dicker
	\item Groundpins mit Verbindung zur Groundplate richtig einstellen
	\item Nach dem Löten mit Messgerät schauen, ob Verbindungen da sind wo sie sein sollten und nicht da sind wo sie nicht sein sollten
\end{itemize}
\section{Software}
	
\section{Schwierigkeiten}
\begin{itemize}
	\item Ground nicht angeschlossen
	\item Stromstecker vergessen
	\item Fehler ohne Debugger zu finden -> Lösungsansatz mit Oszilloskop oder an bestimmten Stellen vom Code LED zum leuchten bringen (um auch zu sehen ob man bestimmte Codeblöcke erreicht - in der richtigen Reihenfolge erreicht)
	\item AtmelStudio STK 500 hinzufügen - allgemein einen funktionierenden Programmer haben - Diamex dann auch mit richtigen Switches auf ON 
\end{itemize}

\end{document}
